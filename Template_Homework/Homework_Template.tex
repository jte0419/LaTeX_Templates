\documentclass[letterpaper,12pt]{article}

% Load packages
\usepackage[left=1.5in,top=1in]{geometry}
\usepackage{graphicx}							% https://ctan.org/pkg/graphicx?lang=en
\usepackage{float}								% https://ctan.org/pkg/float
\usepackage{amsmath}							% https://ctan.org/pkg/amsmath
\usepackage{enumitem}							% https://ctan.org/pkg/enumitem
\usepackage{mathtools}							% https://ctan.org/pkg/mathtools
\usepackage{color}								% https://ctan.org/pkg/color
\usepackage{soul}								% https://ctan.org/pkg/soul
\usepackage{microtype}							% https://ctan.org/pkg/microtype

% Misc
\graphicspath{{./Figures/}}						% Location folder of all figures in paper
\restylefloat{figure}

\begin{document}

% Create header of document
\begin{center}									% Center the header
\textbf{Class Name: Homework 1} \\				% Class name and homework number
\textbf{Thermodynamics Review} 	\\				% Homework name
\textbf{Assigned: 10/32/21~~~~~~ Due: 10/33/21}	% Assigned and due dates
\end{center}

% ===================
% ==== PROBLEM 1 ====
% ===================

\noindent \textul{\textbf{Problem 1}} \\

\noindent Start with the following equation for the 1D energy equation.

\begin{equation}
	h_1 + \frac{1}{2} u_1^2 + q = h_2 + \frac{1}{2} u_2^2
\end{equation}

\noindent For an adiabatic flow, show that the stagnation temperature is constant ($T_{01} = T_{02}$).  What other assumption do you need to make to get to this conclusion? \\

% ===================
% ==== PROBLEM 2 ====
% ===================

\noindent \textul{\textbf{Problem 2}} \\

\noindent Start with the equation defining specific enthalpy.  Assuming a calorically perfect gas, derive the following equation.  Show all steps.

\begin{equation}
	C_p - C_v = R
\end{equation}

\noindent Derive the equation for the specific heat at constant pressure as a function of specific heat ratio and specific gas constant.  Do the same for specific heat at constant volume.

\begin{equation}
\begin{aligned}
	C_p &= C_p\left(\gamma,R\right) \\
	C_v &= C_v\left(\gamma,R\right) \\
\end{aligned}
\end{equation}

\noindent Use the following equation and the equation derived above.

\begin{equation}
	\gamma = \frac{C_p}{C_v}
\end{equation}

% ===================
% ==== PROBLEM 3 ====
% ===================

\noindent \textul{\textbf{Problem 3}} \\

\noindent The boundary work during a reversible process can be derived as:

\begin{equation}
	W_{b,12} = \int_1^2{Pdv}
\end{equation}

\noindent To solve this integral, you need to know the functional relationship between $P$ and $v$ ($P = f(v)$), which is simply the equation of a line on a P-v diagram.  The expression for a curve on the P-v diagram for a polytropic process is the following, where $C$ is a constant.

\begin{equation}
	Pv^n = C
\end{equation}

\noindent When the process is isentropic, the exponent is equal to the ratio of specific heats.  Assume a calorically perfect gas.

\begin{equation}
\begin{aligned}
	n = \gamma & = \frac{C_p}{C_v} \\
	Pv^{\gamma} &= C
\end{aligned}
\end{equation}

\begin{enumerate}[label=(\alph*)]
\item Using the definition of boundary work and the relationship between $P$ and $v$ for an isentropic process, derive the expression for boundary work, seen below.
\begin{equation}
	W_{b,12} = -C_v\left(T_2-T_1\right)
\end{equation}
\item Discuss the resulting sign of the equation derived in part (a) for a compression process, and then for an expansion process.  Does it make sense?
\item The net work over the Brayton cycle can be found by integrating each individual process (e.g. integrate from state 2 to state 3), and then summing them together.
\begin{equation}
	W_{b,total} = W_{b,12} + W_{b,23} + W_{b,34} + W_{b,41}
\end{equation}
Assume air (perfect gas) with $R = 287 \frac{J}{kg \cdot K}$ and $\gamma = 1.4$.  Use the values given below to solve for the net work per unit mass over the cycle.
\begin{equation}
\begin{aligned}
	P_1 &= 26,500~\mathrm{Pa}~~~~~~~~~~& T_1 &= 223~\mathrm{K} \\
	P_2 &= 357,750~\mathrm{Pa}~~~~~~~~~~& T_2 &= 660~\mathrm{K} \\
	P_3 &= 357,750~\mathrm{Pa}~~~~~~~~~~& T_3 &= 1400~\mathrm{K} \\
	P_4 &= 26,500~\mathrm{Pa}~~~~~~~~~~& T_4 &= 900~\mathrm{K} \\
\end{aligned}
\end{equation}
\end{enumerate}


\end{document}